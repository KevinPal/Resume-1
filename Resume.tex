\documentclass[letterpaper,12pt]{article}
\usepackage{multicol}
\usepackage{titlesec}
\usepackage{titling}
\usepackage[margin=0.5in]{geometry}
\usepackage{listings}
\usepackage[pages=some]{background}
\usepackage{enumitem}
\usepackage[nodayofweek, level]{datetime}
%\backgroundsetup{
%scale=1,
%color=black,
%opacity=0.0,
%angle=0,
%contents={%
%\includegraphics[width=\paperwidth,height=\paperheight]{WoodOverlayA4.png}
%}%
%}
\titleformat{\section}{\vspace{-2mm}\huge\bfseries}{}{0em}{}[\titlerule]
\titleformat{\subsection}{\large\bfseries}{}{0em}{}
\titleformat{\subsubsection}[runin]{}{}{0em}{}[ --- ]
\newcommand\textbox[1]{%
	\parbox{.333\textwidth}{#1}%
}
\renewcommand{\maketitle}{
	\begin{center}
		\noindent\textbox{\hfill}\textbox{\hfil\bfseries\huge Kevin Zheng\hfil}\textbox{\hfill \today}
		{zhengkevin256@gmail.com --- (669) 350-9233 --- github.com/coffeevector --- coffeevector.github.io}
	\end{center} }
\pagenumbering{gobble}
\begin{document}
\maketitle
\section{Education}
Evergreen Valley High School\null\hfill (2014 - 2018)\\
\emph{High School Diploma}\null\hfill AP Computer Science A, AP Statistics, AP Physics C: Mech \& EM\\~\\
Evergreen Valley College\null\hfill (2016 - 2018)\\
\null\hfill Calculus II, Multivariable Calculus, Linear Algebra\\~\\
University of California, Merced\null\hfill(2018 - \emph{Expected} May 2021)\newline
\emph{Bachelor of Science, Computer Science and Engineering}\null\hfill Data Structures, Introductory Physics II
\section{Technical Skills}
\subsection{Languages}
Expert in Java, proficient in C\verb!++!, Python, GoLang, Kotlin, and Javascript
\subsection{Operating Systems}
Linux, Windows, Android
\vspace*{-2mm}
\subsection{Other}
git, vim, bash/zsh, ReactJS
\section{Projects}
	\subsection{SPARK Final Project --- 2018 \null\hfill Merced, CA}
	\par Made data analysis code that interpreted oscilloscope data to find the speed of light.
	\subsection{EVHS Robotics (FIRST Robotics) --- 2018 \null\hfill San Jose, CA}
	\par Contributed to computer vision code in C++ and Java that automatically detected and handled yellow cubes on a playing field.
	\subsection{AP Computer Science A Final Project --- 2018 \null\hfill San Jose, CA}
	\par Created a horror game centered around math with a group. Was responsible for generating math problems that included arithmetic, derivatives, integrals, etc., during runtime. Project was written in Java.
	\subsection{Google foobar --- 2018 \null\hfill San Jose, CA}
	\par Completed all levels (1 - 5) of the Google foobar coding challenge twice.
	My solutions had been written in Java with occasional Python.
\end{document}
