\documentclass[letterpaper,12pt]{article}
\usepackage{multicol}
\usepackage{vwcol}
\usepackage{titlesec}
\usepackage{titling}
\usepackage[margin=0.5in]{geometry}
\usepackage{listings}
\usepackage{enumitem}
\usepackage[nodayofweek, level]{datetime}
\titleformat{\section}{\vspace{-2mm}\huge\bfseries}{}{0em}{}[\titlerule]
\titleformat{\subsection}{\large\bfseries}{}{0em}{}
\titleformat{\subsubsection}[runin]{}{}{0em}{}[ --- ]
\newcommand\textbox[1]{%
	\parbox{.333\textwidth}{#1}%
}
\renewcommand{\maketitle}{
	\begin{center}
		\noindent\textbox{\hfill}\textbox{\hfil\bfseries\huge Kevin Zheng\hfil}\textbox{\hfill \today}
		{zhengkevin256@gmail.com --- (669) 350-9233 --- github.com/coffeevector --- coffeevector.github.io}
	\end{center} }
\pagenumbering{gobble}
\begin{document}
\maketitle
\section{Education}
Evergreen Valley College\null\hfill (2016 - 2018)\\
\null\hfill Calculus II, Multivariable Calculus, Linear Algebra\\~\\
University of California, Merced\null\hfill(2018 - \emph{Expected} May 2021)\\
\emph{Bachelor of Science, Computer Science and Engineering}\null\hfill Data Structures, Introductory Physics II
\section{Technical Skills}
	\subsection{Programming Languages}
	Expert in Java and C/C\verb!++!, proficient in Python, GoLang, and Kotlin
	\vspace*{-2mm}
	\subsection{Web Development}
	React, Javascript, Google Analytics, HTML, and CSS
	\vspace*{-2mm}
	\subsection{Other}
	Linux, git, statistics, Android Studio, vim, bash/zsh, {\LaTeX}
\section{Projects}
	\subsection{SPARK Final Project --- 2018 \null\hfill Merced, CA}
	\par Made data analysis code that interpreted oscilloscope data to find the speed of light in Python.
	\vspace*{-2mm}
	\subsection{EVHS Robotics (FIRST Robotics) --- 2018 \null\hfill San Jose, CA}
	\par Contributed to computer vision code in C++ and Java that automatically detected and handled yellow cubes on a playing field.
	\vspace*{-2mm}
	\subsection{AP Computer Science A Final Project --- 2018 \null\hfill San Jose, CA}
	\par Created a horror game centered around math with a group. Was responsible for generating math problems that included arithmetic, derivatives, integrals, etc., during runtime. Project was written in Java.
	\vspace*{-2mm}
	\subsection{Google foobar --- 2018 \null\hfill San Jose, CA}
	\par Completed all levels (1 - 5) of the Google foobar coding challenge twice and had been reached out by Google on both occasions.
	\vspace*{-2mm}
	My solutions had been written in Java with occasional Python.
\section{Awards and Certifications}
	{\bfseries AMATYC} Student Mathematics League --- Certificate of Merit
\end{document}
